\documentclass[submit]{../harvardml}


\course{CS1810-S25}
\assignment{Assignment \#1}
\duedate{11:59pm ET, February 14, 2025} 

\usepackage[OT1]{fontenc}
\usepackage[colorlinks,citecolor=blue,urlcolor=blue]{hyperref}
\usepackage{graphicx}
\usepackage{caption}
\usepackage{enumitem}
\usepackage{soul}
\usepackage{amsmath}
\usepackage{amssymb}
\usepackage{color}
\usepackage{todonotes}
\usepackage{listings}
\usepackage{../common}
\usepackage{framed}
\usepackage{float}
\usepackage{ifthen}
\usepackage{bm}

\usepackage[mmddyyyy,hhmmss]{datetime}

\definecolor{verbgray}{gray}{0.9}

\lstnewenvironment{csv}{
  \lstset{backgroundcolor=\color{verbgray},
  frame=single,
  framerule=0pt,
  basicstyle=\ttfamily,
  columns=fullflexible}}{}

 \DeclareMathOperator*{\limover}{\overline{lim}}

%%%%%%%%%%%%%%%%%%%%%%%%%%%%%%%%%%
%% Solution environment
\usepackage{xcolor}
\usepackage{comment}
\newenvironment{solution}
  {\color{blue}\section*{Solution}}
{}
%%%%%%%%%%%%%%%%%%%%%%%%%%%%%%%%%%



\begin{document}
\begin{center}
  {\Large Homework 1: Regression}\\
\end{center}

\subsection*{Introduction}
This homework is on different three different forms of regression:
kernelized regression, nearest neighbors regression, and linear
regression.  We will discuss implementation and examine their
tradeoffs by implementing them on the same dataset, which consists of
temperature over the past 800,000 years taken from ice core samples.

The folder \verb|data| contains the data you will use for this
problem. There are two files:
\begin{itemize}
  \item \verb|earth_temperature_sampled_train.csv|
  \item \verb|earth_temperature_sampled_test.csv|
\end{itemize}

Each has two columns.  The first column is the age of the ice core
sample.  The second column is the approximate difference in a year's temperature (K)
from the average temperature of the 1,000 years preceding it. The temperatures were retrieved from ice cores in
Antarctica (Jouzel et al. 2007)\footnote{Retrieved from
  \url{https://www.ncei.noaa.gov/pub/data/paleo/icecore/antarctica/epica_domec/edc3deuttemp2007.txt}

  Jouzel, J., Masson-Delmotte, V., Cattani, O., Dreyfus, G., Falourd,
  S., Hoffmann, G., … Wolff, E. W. (2007). Orbital and Millennial
  Antarctic Climate Variability over the Past 800,000 Years.
  \emph{Science, 317}(5839), 793–796. doi:10.1126/science.1141038}.

The following is a snippet of the data file:

\begin{csv}
  # Age, Temperature
  399946,0.51
  409980,1.57
\end{csv}

\noindent And this is a visualization of the full dataset:
\begin{center}
  \includegraphics[width=.8\textwidth]{img_input/sample_graph}
\end{center}
\noindent


\textbf{Due to the large magnitude of the years, we will work in terms
  of thousands of years BCE in these problems.} This is taken care of
for you in the provided notebook.






\subsection*{Resources and Submission Instructions}
If you find that you are having trouble with the first couple
problems, we recommend going over the fundamentals of linear algebra
and matrix calculus (see links on website).  The relevant parts of the
\href{https://github.com/harvard-ml-courses/cs181-textbook/blob/master/Textbook.pdf}{cs181-textbook
  notes are Sections 2.1 - 2.7}.  We strongly recommend reading the
textbook before beginning the homework.

We also encourage you to first read the
\href{http://users.isr.ist.utl.pt/~wurmd/Livros/school/Bishop\%20-\%20Pattern\%20Recognition\%20And\%20Machine\%20Learning\%20-\%20Springer\%20\%202006.pdf}{Bishop
  textbook}, particularly: Section 2.3 (Properties of Gaussian
Distributions), Section 3.1 (Linear Basis Regression), and Section 3.3
(Bayesian Linear Regression). (Note that our notation is slightly
different but the underlying mathematics remains the same!).

\textbf{Please type your solutions after the corresponding problems
  using this \LaTeX\ template, and start each problem on a new page.}
You may find the following introductory resources on \LaTeX\ useful:
\href{http://www.mjdenny.com/workshops/LaTeX_Intro.pdf}{\LaTeX\ Basics}
and
\href{https://www.overleaf.com/learn/latex/Free_online_introduction_to_LaTeX_(part_1)}{\LaTeX\ tutorial
  with exercises in Overleaf}

Homeworks will be submitted through Gradescope. You will be added to
the course Gradescope once you join the course Canvas page. If you
haven't received an invitation, contact the course staff through Ed.

\textbf{Please submit the writeup PDF to the Gradescope assignment
  `HW1'.} Remember to assign pages for each question.

\textbf{Please submit your \LaTeX file and code files to the
  Gradescope assignment `HW1 - Supplemental'.} Your files should be
named in the same way as we provide them in the repository,
e.g. \texttt{hw1.pdf}, etc.

%%%%%%%%%%%%%%%%%%%%%%%%%%%%%%%%%%%%%%%%%%%%%
% Problem 1
%%%%%%%%%%%%%%%%%%%%%%%%%%%%%%%%%%%%%%%%%%%%%
\begin{problem}[kNN and Kernels, 35pts]

You will now implement two non-parametric regressions to model temperatures over time.  
% For this problem, you will use the \textbf{same dataset as in Problem 1}.

\vspace{0.5cm}
\noindent\emph{Make sure to include all required plots in your PDF. Passing all test cases does not guarantee that your solution is correct, and we encourage you to write your own. }

\begin{enumerate}
\item 
 Recall that kNN uses a predictor of the form
\[
  f(x^*) = \frac{1}{k} \sum_n y_n \mathbb{I}(x_n \texttt{ is one of k-closest to } x^*),
\]
where $\mathbb{I}$ is an indicator variable. 
\begin{enumerate}

  \item The kNN implementation \textbf{has been provided for you} in the notebook. Run the cells to plot the results for $k=\{1, 3, N-1\}$, where $N$ is the size of the dataset. Describe how the fits change with $k$. Please include your plot in your solution PDF.

  \item Now, we will evaluate the quality of each model \emph{quantitatively} by computing the error on the provided test set. Write Python code to compute test MSE for each value of $k$.  Which solution has the lowest MSE? 
  
\end{enumerate}

\item \textit{Kernel-based regression} techniques are another form of non-parametric regression. Consider a kernel-based
regressor of the form 
\begin{equation*}
  f_\tau(x^*) = \cfrac{\sum_{n} K_\tau(x_n,x^*) y_n}{\sum_n K_\tau(x_n, x^*)}
\end{equation*}
where $\mathcal{D}_\texttt{train} = \{(x_n,y_n)\}_{n = 1} ^N$ are the
training data points, and $x^*$ is the point for which you want to
make the prediction.  The kernel $K_\tau(x,x')$ is a function that
defines the similarity between two inputs $x$ and $x'$. A popular
choice of kernel is a function that decays as the distance between the
two points increases, such as
\begin{equation*}
  K_\tau(x,x') = \exp\left(-\frac{(x-x')^2}{\tau}\right)
\end{equation*}

where $\tau$ represents the square of the lengthscale (a scalar value that
dictates how quickly the kernel decays).  


\begin{enumerate}
    
  \item First, implement the \texttt{kernel\_regressor} function in the notebook, and plot your model for years in the range $800,000$ BC to $400,000$ BC at $1000$ year intervals for the following three values of $\tau$: $1, 50, 2500$. Since we're working in terms of thousands of years, this means you should plot $(x, f_\tau(x))$ for $x = 400, 401, \dots, 800$. \textbf{In no more than 10 lines}, describe how the fits change with $\tau$. Please include your plot in your solution PDF.

  \item Denote the test set as $\mathcal{D}_\texttt{test} = \{(x'_m, y'_m)\}_{m = 1} ^M$.  Write down the expression for MSE of $f_\tau$ over the test set as a function of the training set and test set. Your answer may include $\{(x'_m, y'_m)\}_{m = 1} ^M$, $\{(x_n, y_n)\}_{n = 1} ^N$, and $K_\tau$, but not $f_\tau$.

    \item Compute the MSE on the provided test set for the three values of $\tau$.  Which model yields the lowest MSE? Conceptually, why is this the case? Why would choosing $\tau$ based on $\mathcal{D}_\texttt{train}$ rather than $\mathcal{D}_\texttt{test}$ be a bad idea? 

  \item Describe the time and space complexity of both kernelized regression and kNN with respect to the size of the training set $N$.  How, if at all, does the size of the model---everything that needs to be stored to make predictions---change with the size of the training set $N$?  How, if at all, do the number of computations required to make a prediction for some input $x^*$ change with the size of the training set $N$?.
  

  \item  What is the exact form of $\lim_{\tau \to 0 }f_\tau(x^*)$?
  \end{enumerate}
\end{enumerate}
\end{problem}

\newpage
\begin{solution}
	Your solution here.
\end{solution}


%%%%%%%%%%%%%%%%%%%%%%%%%%%%%%%%%%%%%%%%%%%%%
% Problem 2
%%%%%%%%%%%%%%%%%%%%%%%%%%%%%%%%%%%%%%%%%%%%%
\newpage
\begin{problem}[Deriving Linear Regression, 20pts]

We now seek to model the temperatures with a parametric method: linear regression. Before we implement anything, let's revisit the mathematical formulation of linear regression.  Specifically, the solution for the least squares linear regression  ``looks'' kind of like a ratio of covariance and
variance terms.  In this problem, we will make that connection more
explicit. \\

\noindent Suppose we have some 2-D data where each observation has the form $(x, y)$ and is independent and identically distributed according  $x \sim p(x)$, $y \sim p(y|x)$. We will consider the process of fitting these data from this distribution with the best linear model
possible, that is a linear model of the form $\hat{y} = wx$ that
minimizes the expected squared loss $E_{x,y}[ ( y - \hat{y} )^2
    ]$.\\

\noindent Note: The notation $E_{x, y}$ indicates an
expectation taken over the joint distribution $p(x,y)$. This essentially just means to treat $x$ and $y$ as random.  

\begin{enumerate}

  \item Derive an expression for the optimal $w$, that is, the $w$
        that minimizes the expected squared loss above.  You should leave
        your answer in terms of moments of the distribution, e.g. terms
        like $E_x[x]$, $E_x[x^2]$, $E_y[y]$, $E_y[y^2]$, $E_{x,y}[xy]$
        etc.

  \item Note that while $x, y$ are data that we have access to, $E_{x, y}[yx]$ is a theoretical constant. Keeping in mind the interpretation of expectations as average values, how could you use observed data $\{(x_n,y_n)\}_{n=1}^N$ to estimate $E_{x, y}[yx]$ and $E_x[x^2]$?

  \item In general, moment terms like $E_{x, y}[yx]$, $E_{x, y}[x^2]$,
        $E_{x, y}[yx^3]$, $E_{x, y}[\frac{x}{y}]$, etc. can easily be
        estimated from the data (like you did above).  If you substitute in
        these empirical moments, how does your expression for the optimal
        $w^*$ in this problem compare with the optimal $\bm{\hat w}$ from Problem 4.3 of HW0?

  \item Many common probabilistic linear regression models assume that
        variables $x$ and $y$ are jointly Gaussian.  Did any of your above
        derivations rely on the assumption that $x$ and $y$ are jointly
        Gaussian?  Why or why not?
\end{enumerate}
\end{problem}

\newpage
\begin{solution}
	Your solution here.
\end{solution}

%%%%%%%%%%%%%%%%%%%%%%%%%%%%%%%%%%%%%%%%%%%%%
% Problem 3
%%%%%%%%%%%%%%%%%%%%%%%%%%%%%%%%%%%%%%%%%%%%%
\newpage
\begin{problem}[Basis Regression, 30pts]

 Having reviewed the theory, we now implement some linear regression models for the temperature. If we just directly use the data as given to us, we would only have
    a one dimensional input to our model, the year.  To create a more expressive linear
    model, we will introduce basis functions.

\vspace{1em}

\noindent\emph{Make sure to include all required plots in your PDF.}

\begin{enumerate}
  \item
        We will first implement the four basis regressions below. (The first basis has been implemented for you in the notebook as an example.) Note that we introduce an addition transform $f$ (already into the provided notebook) to address concerns about numerical instabilities.
        \begin{enumerate}
          \item $\phi_j(x)= f(x)^j$ for $j=1,\ldots, 9$. $f(x) = \frac{x}{1.81 \cdot 10^{2}}.$
          \item $\phi_j(x) = \exp\left\{-\cfrac{(f(x)-\mu_j)^2}{5}\right\}$ for $\mu_j=\frac{j + 7}{8}$ with $j=1,\ldots, 9$. $f(x) = \frac{x}{4.00 \cdot 10^{2}}.$
          \item $\phi_j(x) =  \cos(f(x) / j)$ for $j=1, \ldots, 9$. $f(x) = \frac{x}{1.81}$.
          \item $\phi_j(x) = \cos(f(x) / j)$ for $j=1, \ldots, 49$. $f(x) = \frac{x}{1.81 \cdot 10^{-1}}$. \footnote{For the trigonometric bases (c) and (d), the periodic nature of
                  cosine requires us to transform the data such that the
                  lengthscale is within the periods of each element of our basis.}
        \end{enumerate}

        {\footnotesize * Note: Please make sure to add a bias term for
        all your basis functions above in your implementation of the
        \verb|make_basis|.}

        Let
        $$ \mathbf{\phi}(\mathbf{X}) =
          \begin{bmatrix}
            \mathbf{\phi}(x_1) \\
            \mathbf{\phi}(x_2) \\
            \vdots             \\
            \mathbf{\phi}(x_N) \\
          \end{bmatrix} \in \mathbb{R}^{N\times D}.$$
        You will complete the \verb|make_basis| function which must return
        $\phi(\mathbf{X})$ for each part
        (a) - (d). You do NOT need to submit this
        code in your \LaTeX writeup.

        Then, create a plot of the fitted
        regression line for each basis against a scatter plot
        of the training data. Boilerplate plotting code is provided in the
        notebook---you will only need to finish up a part of it.
        \textbf{All you need to include
          in your writeup for this part are these four plots.}

        \item
          Now we have trained each of our basis regressions. For each basis
          regression, compute the MSE on the test set.  Discuss: do any of the
          bases seem to overfit?  Underfit?  Why?


    \item Briefly describe what purpose the transforms $f$ serve: why are they helpful?

    \item As in Problem 1, describe the space and time complexity of linear regression.  How does what is stored to compute predictions change with the size of the training set $N$ and the number of features $D$?  How does the computation needed to compute the prediction for a new input depend on the size of the training set $N$?  How do these complexities compare to those of the kNN and kernelized regressor?

    \item Briefly compare and constrast the different regressors: kNN,
          kernelized regression, and linear regression (with bases).  Are some
          regressions clearly worse than others?  Is there one best
          regression?  How would you use the fact that you have these multiple
          regression functions?

  \end{enumerate}
  Note:
  Recall that we are using a
  different set of inputs $\mathbf{X}$ for each basis (a)-(d).
  Although it may seem as though this prevents us from being able
  to directly compare the MSE since we are using different data,
  each transformation can be considered as being a part of our model.
  Contrast this with transformations (such as standardization) that cause the variance of the target $\mathbf{y}$ to be different; in these cases the
  MSE can no longer be directly compared.
\end{problem}

\newpage 
\begin{solution}
	Your solution here.
\end{solution}

%%%%%%%%%%%%%%%%%%%%%%%%%%%%%%%%%%%%%%%%%%%%%
% Problem 4
%%%%%%%%%%%%%%%%%%%%%%%%%%%%%%%%%%%%%%%%%%%%%
\begin{problem}[Probablistic Regression and Regularization, 30pts]

Finally, we will preview Bayesian regression and explore its connection to regularization for linear models. Then, we will fit a regularized model to the temperature data. Although the content is related, you do not need to know the material from the lectures on frequentist model selection and Bayesian model selection to solve this problem.  \\

\noindent Recall that the probabilistic version of linear regression states that 
\[y_n = \boldw^\top\boldx_n + \epsilon_n, \quad \epsilon_n \sim \mathcal{N}(0, \sigma^2)\]
In Bayesian regression, we impose a prior $p(\boldw)$ on the weights and  fit the weights $\boldw$ through maximizing the posterior likelihood
\[p(\boldw | \boldX, \boldy) = \frac{p(\bold y | \boldw, \boldX)p(\boldw)}{p(\boldy | \boldX)}\]
Note: since we maximize with respect to $\boldw$, it suffices to just maximize the numerator.

\begin{enumerate}
    \item Suppose $\boldw \sim \mathcal{N}(\mathbf{0},\frac{\sigma^2}{\lambda}\boldI)$. Show that maximizing the posterior likelihood is equivalent to minimizing 
    \[\mathcal{L}_{ridge}(\boldw) = \frac{1}{2}||\boldy -\bold X\boldw||_2^2 + \frac{\lambda}{2}||\boldw||_2^2.\] 
    Note that minimizing $\mathcal{L}_{ridge}(\boldw)$ is exactly what ridge regression does.
    
    Hint: You don't need to solve for the maximizer/minimizer to show that the optimization problems are equivalent.
    
    \item Solve for the value of $\boldw$ that minimizes $\mathcal L_{ridge}(\boldw)$.

    \item The Laplace distribution has the PDF
   \[L(a,b) =\frac{1}{2b} \exp\left(-\frac{|x - a|}{b}\right)\]
Show that if all $w_d \sim L\left(0,\frac{2\sigma^2}{\lambda}\right)$, maximizing the posterior likelihood is equivalent to minimizing 
\[\mathcal{L}_{lasso}(\boldw) = \frac{1}{2}||\boldy -\bold X\boldw||_2^2  + \frac{\lambda}{2}||\boldw||_1.\] 
Note that minimizing $\mathcal{L}_{lasso}(\boldw)$ is exactly what LASSO regression does.

    \item Why is there no general closed form for the LASSO estimator, i.e. the value of $\boldw$ that minimizes $\mathcal{L}_{ridge}(\boldw)$?

    \item Since there is no general closed form for LASSO, we use numerical methods for estimating $\boldw$. One approach is to use \textit{coordinate descent}, which works as follows: 
    \begin{enumerate}
        \item Initialize $\boldw=\boldw_0$.
        \item For each $d=1, \ldots, D$ do the following 2 steps consecutively:
        \begin{enumerate}
            \item Compute $\rho_d = \tilde{\boldx}_d^\top(\boldy - (\boldX \boldw - w_d \tilde{\boldx}_d))$. We define $\tilde{\boldx}_d$ as the $d$-th column of $\boldX$.

            \item If $d=1$, set $w_1 = \frac{\rho_1}{||\tilde{\boldx}_1||^2_2}$. Otherwise if $d\ne 1$, compute $w_d = \frac{\text{sign}(\rho_d)\max\left\{|\rho_d|-\frac{\lambda}{2}, 0\right\}}{||\tilde{\boldx}_d||^2_2}$.
        \end{enumerate}
        \item Repeat step (b) until convergence or the maximum number of iterations is reached.
    \end{enumerate} 

    Implement the \texttt{find\_lasso\_weights} function according to the above algorithm, letting $\boldw_0$ be a vector of ones and the max number of iterations be 5000. Then, fit models with $\lambda=1, 10$ to basis (d) from Problem 3, plot the predictions, and compute the MSE's. You will need to do some preprocessing, but a completed helper function for this is already provided. How do the graphs and errors compare to those for the unregularized basis (d) model? 


\end{enumerate}

\end{problem}

\newpage
\begin{solution}
	Your solution here.
\end{solution}


%%%%%%%%%%%%%%%%%%%%%%%%%%%%%%%%%%%%%%%%%%%%%
% Name and Calibration
%%%%%%%%%%%%%%%%%%%%%%%%%%%%%%%%%%%%%%%%%%%%%
\newpage
\subsection*{Name}

\subsection*{Collaborators and Resources}
Whom did you work with, and did you use any resources beyond cs181-textbook and your notes?


\end{document}
